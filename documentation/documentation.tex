\documentclass[10pt,a4paper]{article}
\usepackage[latin1]{inputenc}
\usepackage[margin = 3.5cm,tmargin = 5cm]{geometry}

\usepackage{fancyhdr}
\pagestyle{fancy}
\fancyhf{}
\lhead{Carl Schaffer}
\chead{\ta{} Documentation}
\rhead{Version 1.0}
\rfoot{Page \thepage}

\renewcommand{\headrulewidth}{0.4pt}
\renewcommand{\footrulewidth}{0.4pt}

\usepackage{amsmath}

\usepackage{csquotes}
\usepackage{amsfonts}
\usepackage{amssymb}
\usepackage{graphicx}
\usepackage{amsthm}
\usepackage{color}

\let\oldv\verbatim
\let\oldendv\endverbatim

\def\verbatim{\par\setbox0\vbox\bgroup\oldv}
\def\endverbatim{\oldendv\egroup\fboxsep0pt \noindent\colorbox[gray]{0.8}{\usebox0}\par}



\title{Documentation for C${++}$ \LaTeX{} assistant ZetoTex}
\date{\today{}}
\author{Carl Schaffer}
        




\begin{document}

      \maketitle
      \newpage{}

      % Writing helpers
      \setlength{\parindent}{0pt}

      \newcommand{\stdv}{\texttt{std::vector} } 
      \newcommand{\la}{LaTeX Assistant}
      \newcommand{\ta}{\texttt{\textbf{tex\_assistant}}}
      \newcommand{\filename}[1]{\texttt{#1}}
      \newcommand{\cpp}{\textbf{C++}}
      \newcommand{\ztt}{ZetoTex}

      \newcommand{\singleValCmd}{\cmdstyle{.ncmdValue()}}
      \newcommand{\arrayCmd}{\cmdstyle{.ncmdArray()}}
      \newcommand{\cmdstyle}[1]{\textbf{\texttt{#1}}}

      \newcommand{\headerfile}{\filename{zetotex.hpp}}
      \newcommand{\texfile}{\enquote{.tex} file}


\section{Introduction}
%TexAssistant file written by Tex Assistant
<<<<<<< HEAD:Scripts/zetotex/output/tex_sample.tex
%Written on: 28.04.2015 at 01:56
=======
%Written on: 24.04.2015 at 11:08
>>>>>>> c65d6c035af517cc9d3245acd7f3b0316d06ad37:Scripts/latex_assistant/output/tex_sample.tex


\newcommand{\commandWithFloat}{1.0002}

\newcommand{\commandWithInteger}{1}

\newcommand{\SampleArray}{
	\begin{tabular}{|c|c|c|c|c|c|}
	\hline Column 0 & Column 1 & Column 2 & Column 3 & Column 4 & Column 5 \\
	\hline 0 & 1 & 2 & 3 &   &  \\
	\hline 0 & 1 & 2 & 3 &   &  \\
	\hline 0 & 1 & 2 & 3 &   &  \\
	\hline 0 & 1 & 2 & 3 &   &  \\
	\hline
 	\end{tabular}
}

 \ztt{} provides an assistant to simplify
getting analysis results into finished \LaTeX{} documents. It
provides a custom class designed to convert data objects inherent to \cpp{} into
\LaTeX{} code. This is done by a \la{} which receives and processes the data
contents and produces a \enquote{.tex} file providing custom commands usable for
writing. This document will show how to use the Latex assistant and provide
examples of output.


\section{Using the Assistant}
In order to use the assistant, the custom header \headerfile{} needs to be
included in your \cpp{}-project. this can be done using
\begin{verbatim}
#include "path/to/zetotex.h"
\end{verbatim}
in your code header. This provides the \ta{} class which will be used to handle
all actions described in the following sections.
\subsection{The \enquote{tex\_assistant} Class}

The interface consists of an object of the type \ta{}. The constructor takes one
argument, which is a string containing the path where the latex document
containing the new commands should be created. An empty version of this file is
recreated every time the \ta{} constructor is called. The actual writing is done
by calling methods inherent to the \ta{} class. Example Constructor call:

\begin{verbatim}
tex_assistant ta = tex_assistant("./output/tex_sample.tex");
\end{verbatim}

\subsection{Single Number Commands}

A command to represent a single number can be generated using the
\singleValCmd{} method. It can be called as:
\begin{verbatim}
ta.ncmdValue("commandWithFloat",1.002);
\end{verbatim}
for float values or as
\begin{verbatim}
ta.ncmdValue("commandWithInteger",1);
\end{verbatim}
which produces the following output in the \enquote{.tex} file:
\begin{verbatim}
\newcommand{\commandWithFloat}{1.0002}

\newcommand{\commandWithInteger}{1}
\end{verbatim}

Invoking these commands in a document yields:

\begin{center}
  \begin{tabular}{l|l}
    Command & Output\\ \hline{}
    \verb|\commandWithFloat| & \commandWithFloat\\
    \verb|\commandWithInteger| & \commandWithInteger\\
  \end{tabular}
\end{center}

\newpage{}
\subsection{Writing arrays using ncmdArray}
Data available in \stdv{} form can be written to using \arrayCmd{}. This command
requires three arguments:

\begin{enumerate}
\item the command name as a string
\item a \stdv{} containing the column headers
\item a \stdv{}\texttt{(}\stdv\texttt{)} containing single \stdv{}'s for each
  line
\end{enumerate}

Here is some sample output from first calling
\begin{verbatim}
  ta.ncmdArray("SampleArray",header,data);
\end{verbatim}
which produces the following in the \texfile{:

\begin{verbatim}
\newcommand{\SampleArray}{
        \begin{tabular}{|c|c|c|c|c|c|}
          \hline Column 0 & Column 1 & Column 2 & Column 3 & Column 4 & Column 5 \\
          \hline 0 & 1 & 2 & 3 &   &  \\
          \hline 0 & 1 & 2 & 3 &   &  \\
          \hline 0 & 1 & 2 & 3 &   &  \\
          \hline 0 & 1 & 2 & 3 &   &  \\
          \hline
        \end{tabular}
}

\end{verbatim}

  and then calling \cmdstyle{\tt \string \SampleArray} within a table
  environment in the final \LaTeX{} document.
  \begin{table}[htbp]
    \centering
    \SampleArray
    \caption{Sample array}
    \label{tab:sample1}
  \end{table}

  Note that in this example, the \stdv{} containing the column header strings is
  larger than the data. In this case, the assistant automatically fills empty
  cells to the right of the existing data after printing a warning on the
  console.

\end{document}